\documentclass{report}
\usepackage{hyperref}
\usepackage{listings}
\usepackage{graphicx} % Required for inserting images

\title{Computer Architecture WS2324}
\author{Thierry Meiers}
\date{\today}

\begin{document}

\maketitle

\chapter*{Exercise Sheet 2}
\section*{Exercise 1}
\begin{enumerate}
    \item
    {
        For the following C statement, write the corresponding RISC-V assembly code using a minimal number of instructions. Notice that the addi instruction can also contain negative operands!
        \begin{lstlisting}{language=C}
        f = g + (h - 5);
        \end{lstlisting}
        \textbf{Solution:}
        \href{https://stackoverflow.com/a/61046459}{Hint}
        \begin{lstlisting}{language=RISC-V}
        addi x5 x6 -5       # x5 = h - 5
        add x5 x7 x5        # x5 = g + x5
        \end{lstlisting}
    }
    \item
    {
        Write a single C statement that corresponds to the two RISC-V assembly instructions below.
        \begin{lstlisting}{language=RISC-V}
        add x5, x6, x7
        add x5, x7, x5
        \end{lstlisting}
        \textbf{Solution:}
        \begin{lstlisting}{language=C}
        f = g + (h + i)
        \end{lstlisting}
    }
    \item 
    {
        For the following C statement, write the corresponding RISC-V assembly code using word
        instructions. Assume that the base addresses of the arrays A and B are in registers x10 and
        x11, respectively. E.g. lw x5, 40(x10) would load A[10] into register x5.
        Notice that a word uses 4 bytes in memory, whereas each memory address refers to a single
        byte. So you have to multiply the array indexes by 4.
        \begin{lstlisting}{language=C}
        B[8] = A[i-j];
        \end{lstlisting}
        \textbf{Solution:}
        \begin{lstlisting}{language=RISC-V}
        sub x5, x6, x7      # x5 = i - j
        slli x5, x5, 2      # x5 = (i - j) * 4
        lw x5, 0(x10)       # x5 = A[x5]
        sw x5, 32(x11)      # B[8] = x5        
        \end{lstlisting}
    }
\end{enumerate}

\section*{Exercise 2}
We consider a program running at 6 GHz and we assume that electricity if running a the speed of
light.
\begin{itemize}
    \item
    {
        What size can your CPU have such that a single can go from one size to the other in a cycle?\\
        \textbf{Solution:}\\
        The formula for calculating the distance a signal can travel within a single clock cycle is given by: \[Distance = \frac{Speed_of_light}{Clock_speed} = \frac{3\cdot 10^8 m/s}{6\cdot 10^9Hs} = 0.05m\]
        So, in one clock cycle of a 6 GHz processor, a signal can travel approximately 5 centimeters.
    }
    \item
    {
        Can this problem be improved with 3D CPUs?\\
        \textbf{Solution:}\\
        Yes! With 3D stacking, you can integrate multiple layers of logic, reducing the distance signals need to travel vertically.
    }
\end{itemize}
\end{document}